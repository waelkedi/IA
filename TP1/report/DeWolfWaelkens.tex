\documentclass[10pt,a4paper]{article}
\usepackage{amsmath}
\usepackage{amsfonts}
\usepackage{amssymb}
\usepackage{graphicx}
\usepackage[french]{babel}
\usepackage[T1]{fontenc}
\usepackage[utf8x]{inputenc}
\graphicspath{{images/}}
\usepackage{parskip}
\usepackage{fancyhdr}
\usepackage{vmargin}
\setmarginsrb{3 cm}{2.5 cm}{3 cm}{2.5 cm}{1 cm}{1.5 cm}{1 cm}{1.5 cm}

\title{Détection des feu de forets avec un réseau de neurones embarqué}                             % Title
\author{Maxime De Wolf\\
		Dimitri Waelkens}                               % Author
\date{\today}                                           % Date

\makeatletter
\let\thetitle\@title
\let\theauthor\@author
\let\thedate\@date
\makeatother

\pagestyle{fancy}
\fancyhf{}
\rhead{\theauthor}
\lhead{\thetitle}
\cfoot{\thepage}

\begin{document}
   	
   	%%%%%%%%%%%%%%%%%%%%%%%%%%%%%%%%%%%%%%%%%%%%%%%%%%%%%%%%%%%%%%%%%%%%%%%%%%%%%%%%%%%%%%%%%
   	
   	\begin{titlepage}
   		\centering
   		\vspace*{0.5 cm}
   		\includegraphics[scale = 0.75]{UMONS}\\[1.0 cm]   % University Logo
   		\textsc{\LARGE Université de Mons}\\[2.0 cm]   % University Name
   		\textsc{\large Défi en en Intelligence Artificielle}\\[0.5 cm]               % Course Name
   		\rule{\linewidth}{0.2 mm} \\[0.4 cm]
   		{ \huge \bfseries \thetitle}\\
   		\rule{\linewidth}{0.2 mm} \\[1.5 cm]
   		
   		\begin{minipage}{0.4\textwidth}
   			\begin{flushleft} \large
   				\emph{Auteur:}\\
   				\theauthor
   			\end{flushleft}
   		\end{minipage}~
   		\begin{minipage}{0.4\textwidth}
   			\begin{flushright} \large
   				\emph{Dimitri Waelkens}                                  % Your Student Number
   			\end{flushright}
   		\end{minipage}\\[2 cm]
   		
   		{\large \thedate}\\[2 cm]
   		
   		\vfill
   		
   	\end{titlepage}
   	
   	\section{Objectif}
   	
   		Dans le cadre du cours de \textit{Défi en Intelligence Artificielle}, il nous a été demandé d'entraîner un réseau de neurones afin de pouvoir détecter les feux de forêts grâce à celui-ci.\\
   		Plus précisément, notre réseau de neurones doit être capable de classer des images dans trois catégories distinctes: \texttt{fire, start\_fire, no\_fire}. Ces catégories représentent respectivement les images montrant un incendie, les images montrant un début d'incendie -c'est à dire une image qui contiennent beaucoup de fumée- et les images ne présentant aucune trace d'incendie.
   		
   	\section{Structure du réseau}
   	
   		Comme base de notre réseau, nous avons choisis un réseau de neurones \textbf{Xception} pré-entraîné grâce à \textit{Imagenet}. Nous rajoutons ensuite une couche de \textit{pooling} ainsi qu'une couche \textit{fully-connected}. Ensuite nous entraînons notre réseau de neurones en deux étapes. 
   		
   		La première étape consiste à entraîner les couches que nous avons rajoutées au réseau pré-entraîné. Ensuite nous entraînons l'intégralité du réseau ainsi obtenu.
   	
   	\section{Résultats}
   	
   	\section{Améliorations possibles}
   	
   		Le \textit{dataset} actuel n'est pas de très bonne qualité. En effet, celui-ci ayant été récolté en décomposant des vidéos image par image, cela implique que les données d'entraînement sont fortement liées les unes aux autres de par leur ressemblance. Pour pallier à ce problème, nous aurions pû faire de l'augmentation de données sur le \textit{dataset}. Cela consiste à appliquer une modification aléatoire à l'image (zoom, rotation, filtre, ...) avant de l'utiliser pour entraîner le réseau de neurones. Cela permettrait de "casser" la ressemblance entre deux images et ainsi d'améliorer artificiellement la qualité du \textit{dataset}.
   	
   	\newpage
   	%\bibliographystyle{plain}
   	%\bibliography{biblist}
          	
\end{document}